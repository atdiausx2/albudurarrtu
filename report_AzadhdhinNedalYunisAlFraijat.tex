% Options for packages loaded elsewhere
\PassOptionsToPackage{unicode}{hyperref}
\PassOptionsToPackage{hyphens}{url}
\PassOptionsToPackage{dvipsnames,svgnames,x11names}{xcolor}
%
\documentclass[
  letterpaper,
  DIV=11,
  numbers=noendperiod]{scrartcl}

\usepackage{amsmath,amssymb}
\usepackage{iftex}
\ifPDFTeX
  \usepackage[T2A]{fontenc}
  \usepackage[utf8]{inputenc}
  \usepackage{textcomp} % provide euro and other symbols
\else % if luatex or xetex
  \usepackage{unicode-math}
  \defaultfontfeatures{Scale=MatchLowercase}
  \defaultfontfeatures[\rmfamily]{Ligatures=TeX,Scale=1}
\fi
\usepackage{lmodern}
\ifPDFTeX\else  
    % xetex/luatex font selection
  \setmainfont[]{CMU Serif}
  \setsansfont[]{CMU Serif}
  \setmonofont[]{CMU Serif}
\fi
% Use upquote if available, for straight quotes in verbatim environments
\IfFileExists{upquote.sty}{\usepackage{upquote}}{}
\IfFileExists{microtype.sty}{% use microtype if available
  \usepackage[]{microtype}
  \UseMicrotypeSet[protrusion]{basicmath} % disable protrusion for tt fonts
}{}
\usepackage{xcolor}
\setlength{\emergencystretch}{3em} % prevent overfull lines
\setcounter{secnumdepth}{-\maxdimen} % remove section numbering
% Make \paragraph and \subparagraph free-standing
\ifx\paragraph\undefined\else
  \let\oldparagraph\paragraph
  \renewcommand{\paragraph}[1]{\oldparagraph{#1}\mbox{}}
\fi
\ifx\subparagraph\undefined\else
  \let\oldsubparagraph\subparagraph
  \renewcommand{\subparagraph}[1]{\oldsubparagraph{#1}\mbox{}}
\fi


\providecommand{\tightlist}{%
  \setlength{\itemsep}{0pt}\setlength{\parskip}{0pt}}\usepackage{longtable,booktabs,array}
\usepackage{calc} % for calculating minipage widths
% Correct order of tables after \paragraph or \subparagraph
\usepackage{etoolbox}
\makeatletter
\patchcmd\longtable{\par}{\if@noskipsec\mbox{}\fi\par}{}{}
\makeatother
% Allow footnotes in longtable head/foot
\IfFileExists{footnotehyper.sty}{\usepackage{footnotehyper}}{\usepackage{footnote}}
\makesavenoteenv{longtable}
\usepackage{graphicx}
\makeatletter
\def\maxwidth{\ifdim\Gin@nat@width>\linewidth\linewidth\else\Gin@nat@width\fi}
\def\maxheight{\ifdim\Gin@nat@height>\textheight\textheight\else\Gin@nat@height\fi}
\makeatother
% Scale images if necessary, so that they will not overflow the page
% margins by default, and it is still possible to overwrite the defaults
% using explicit options in \includegraphics[width, height, ...]{}
\setkeys{Gin}{width=\maxwidth,height=\maxheight,keepaspectratio}
% Set default figure placement to htbp
\makeatletter
\def\fps@figure{htbp}
\makeatother

\KOMAoption{captions}{tablesignature}
\usepackage{csquotes}
\usepackage[authordate, abbreviate = true, date = year, autocite=inline, backref = true]{biblatex-chicago}
\makeatletter
\makeatother
\makeatletter
\makeatother
\makeatletter
\@ifpackageloaded{caption}{}{\usepackage{caption}}
\AtBeginDocument{%
\ifdefined\contentsname
  \renewcommand*\contentsname{Table of contents}
\else
  \newcommand\contentsname{Table of contents}
\fi
\ifdefined\listfigurename
  \renewcommand*\listfigurename{List of Figures}
\else
  \newcommand\listfigurename{List of Figures}
\fi
\ifdefined\listtablename
  \renewcommand*\listtablename{List of Tables}
\else
  \newcommand\listtablename{List of Tables}
\fi
\ifdefined\figurename
  \renewcommand*\figurename{Figure}
\else
  \newcommand\figurename{Figure}
\fi
\ifdefined\tablename
  \renewcommand*\tablename{Table}
\else
  \newcommand\tablename{Table}
\fi
}
\@ifpackageloaded{float}{}{\usepackage{float}}
\floatstyle{ruled}
\@ifundefined{c@chapter}{\newfloat{codelisting}{h}{lop}}{\newfloat{codelisting}{h}{lop}[chapter]}
\floatname{codelisting}{Listing}
\newcommand*\listoflistings{\listof{codelisting}{List of Listings}}
\makeatother
\makeatletter
\@ifpackageloaded{caption}{}{\usepackage{caption}}
\@ifpackageloaded{subcaption}{}{\usepackage{subcaption}}
\makeatother
\makeatletter
\@ifpackageloaded{tcolorbox}{}{\usepackage[skins,breakable]{tcolorbox}}
\makeatother
\makeatletter
\@ifundefined{shadecolor}{\definecolor{shadecolor}{rgb}{.97, .97, .97}}
\makeatother
\makeatletter
\makeatother
\makeatletter
\makeatother
\ifLuaTeX
\usepackage[bidi=basic]{babel}
\else
\usepackage[bidi=default]{babel}
\fi
\babelprovide[main,import]{english}
% get rid of language-specific shorthands (see #6817):
\let\LanguageShortHands\languageshorthands
\def\languageshorthands#1{}
\ifLuaTeX
  \usepackage{selnolig}  % disable illegal ligatures
\fi
\usepackage[]{biblatex}
\addbibresource{bibliography.bib}
\usepackage{csquotes}
\IfFileExists{bookmark.sty}{\usepackage{bookmark}}{\usepackage{hyperref}}
\IfFileExists{xurl.sty}{\usepackage{xurl}}{} % add URL line breaks if available
\urlstyle{same} % disable monospaced font for URLs
\hypersetup{
  pdftitle={Student dropout analysis based on previously acquired educational achievements: A case of the University of Portalegre},
  pdfauthor={Izeldeen Nedal Yunis Al Fraijat; Danat Semeneev; Ieva Žube; Pankaj Chettri; Kristaps Eglītis},
  pdflang={en},
  pdfkeywords={AA1, AA2},
  colorlinks=true,
  linkcolor={blue},
  filecolor={Maroon},
  citecolor={Blue},
  urlcolor={Blue},
  pdfcreator={LaTeX via pandoc}}

\title{Student dropout analysis based on previously acquired educational
achievements: A case of the University of Portalegre\thanks{All
correspondence should be addressed to
\href{mailto:danat.semeneev@edu.rtu.lv}{\nolinkurl{danat.semeneev@edu.rtu.lv}}.}}
\usepackage{etoolbox}
\makeatletter
\providecommand{\subtitle}[1]{% add subtitle to \maketitle
  \apptocmd{\@title}{\par {\large #1 \par}}{}{}
}
\makeatother
\subtitle{Business Analytics course, Business Informatics Ms programme,
Fall 2023.}
\author{Izeldeen Nedal Yunis Al Fraijat \and Danat Semeneev \and Ieva
Žube \and Pankaj Chettri \and Kristaps Eglītis\footnote{(all) Rīga
  Technical University}}
\date{2023-12-14}

\begin{document}
\maketitle
\begin{abstract}
In the world of education, a person's academical path is often depicted
as a linear progression, where students follow a predefined journey from
kindergarten to graduation. However, the reality is far more complex.
There could be various reasons that emerge during the study program that
led students to deviate from this path. These students might encounter
different challenges, circumstances, or a lack of proper resources that
have led them to drop out of university. In this dataset provided to us,
we will delve deeper into understanding the reasons why students have
dropped out of the university. We will leverage our social knowledge to
comprehend the factors that influenced their decision to drop out and
work to prevent such occurrences if the issues are within the
university's purview. Our goal is to propose solutions and their support
so that high schools could facilitate and secure students' educational
journeys. We construct the best basic model that predicts the results
with 73\% balanced precision recall. We provide the recommedatons based
on findings and define auxiliary requisite data to further our research
in the future.
\end{abstract}
\ifdefined\Shaded\renewenvironment{Shaded}{\begin{tcolorbox}[interior hidden, boxrule=0pt, sharp corners, frame hidden, borderline west={3pt}{0pt}{shadecolor}, breakable, enhanced]}{\end{tcolorbox}}\fi

\renewcommand*\contentsname{Table of contents}
{
\hypersetup{linkcolor=}
\setcounter{tocdepth}{3}
\tableofcontents
}

\hypertarget{goals-and-objectives}{%
\section{Goals and Objectives}\label{goals-and-objectives}}

In order to shed light on the factors for dropout and the dropout
itself, we have set the following objectives in this research:

\begin{itemize}
\tightlist
\item
  Basing on literature sources, briefly recap incentives towards
  studying and its termination
\item
  Formulate Metrics and Key Performance Indicators (KPIs)

  \begin{itemize}
  \tightlist
  \item
    Define measurable KPIs for effective tracking and pinpointing the
    dropout related trends
  \item
    Establish benchmark for success and areas that requires attention
    for improvement
  \end{itemize}
\item
  Perform Analysis and Pattern~Recognition

  \begin{itemize}
  \tightlist
  \item
    Identify key factors that contribute to the dropout rate of students
  \item
    Using these factors, propose and build a model predicting students'
    dropout from the university, graduation, or further enrollment.
  \item
    Evaluate the model built
  \end{itemize}
\item
  Designing Targeted Interventions

  \begin{itemize}
  \tightlist
  \item
    Based on patterns identified, propose targeted interventions
  \item
    Brainstorm and implement various strategies to address dropout
    factors
  \end{itemize}
\end{itemize}

\hypertarget{introduction}{%
\section{Introduction}\label{introduction}}

Starting from preliminary school we are told that having an education is
very important for your future or that without higher education your job
possibilities are going to be very limited. While primary education is
mandatory, having higher education is not. There are, however, many
reasons for which people may want to pursue higher education. According
to studies, many factors are materialistic, the most important factor
for pursuing higher education is job acquisition
\autocite{knutsen_motivation_2011}. Some other factors may include
increased income in the existing job, improved work conditions or
increased ability for retirement. Of course, other, more intrisic
factors include seeking for additional knowledge or self-fulfillment
\autocite{cortes_factors_2023}. There are also factors like meeting new
friends, improving social interaction skills or just wanting to make a
difference in the world. Of course factors that cannot be ignored are
social pressure \autocite{temple_factors_2009}, meaning that having
friends that want to pursue higher education can influence ones own
decision or influence of family members. However, there are people that
discontinue their studies prematurely and we are interested to learn
what the reasons for such a decision could be. Based on the study and
datasets that we used for our research there are multiple factors that
influence dropping out.

Nevertheless, pursuing higher education and actually getting the degree
has some tangible benefits. According to an OECD -- Education at a
Glance 2019 research paper \autocite{oecd_education_2019}.

\begin{quote}
\enquote{On average across OECD countries, adults with a short-cycle
tertiary degree earn 20\% more than adults with upper secondary
education. The earnings advantage increases to 44\% for those with a
bachelor's degree and to 91\% for those with a master's or doctoral
degree.}
\end{quote}

With this in mind, it is important for government and educational
institutions to ensure high level of graduates in society to ensure
economic growth and overall increase in well-being. To measure the
success of this goal, it is important to set KPI's, track them and make
educated conclusions on what needs to be done or is being done right to
reach the goal of higher educated society.

\hypertarget{sec-kpi}{%
\section{Target Metrics and KPI}\label{sec-kpi}}

In this particular case, KPI's will be chosen based on datasets of
Portugese High Schools but most likely data can be generalised, atleast
for Europe, as the region and sociodemographics are not so different.
Even though there are many factors that influence the success of
graduation, only factors that can be proven by government and
educational institutions will be chosen. In order to thwart
embezzlement, indicators should be restricted in magnitude and difficult
to falsify or manipulate. After rigorous analysis, we propose the
following KPIs.

\begin{enumerate}
\def\labelenumi{\alph{enumi}.}
\tightlist
\item
  \textbf{Student grade improvement compared to support}. Based on the
  dataset, students who had support had 3x lower dropout rates than
  students that didn't have. While it is not practical to allocate
  higher amount of money for studying that itself does not generate
  value, it scoops that it at least a sizeable parts of the dropout
  students could be held from leaving with a relatively small aid that
  would make the benefits of studies outweigh those of working/etc.
  Leaving is commonly associated with very poor grades (otherwise, even
  a morally disinterested student would opt to formally remain in the
  university until they are asked to leave due to poor performance).
  Since a person with infinitesimal grades is a clear candidate for
  dropping out, one should identify those students with abrupt downward
  grade dynamics and quench this. In the proposed KPI, the
  \((grade)_{i}\) is the mean relative grade change for student \(j\)
  over all their courses at university i at moment t, and the assistance
  is the mean aid per student (can be 0). If there are no students on
  their way down , the KPI is guaranteed to be positive.
\end{enumerate}

\[KPI_{1, i,t} =  \frac{|\Delta(\bar{grade})_i|}{ (\bar{assistance}_i)}\]
This does not depend on the number of courses, because the courses are
themselves different difficulties, the important thing that the
university (the students too) should look after in this regard, that the
situation with grades does drastically deteriorate over time.

\begin{enumerate}
\def\labelenumi{\alph{enumi}.}
\item
  \textbf{Institutional Improvements}. Although volatile and subjective,
  as one of the metrics (not KPIs, since it is more difficult to tie
  this to specific redresses) there could be a longitudinal survey about
  one's satisfaction with the studies and programme in general in the
  fashion of a job an exit or quasi-exit interview (when a person does
  not leave actually, but they are still invited to answer the questions
  as if they would be leaving). This would allow to track the scale of
  dropouts due to frustration with the programme (not engaging enough).
\item
  \textbf{Relative changes in student's grades}. Datasets tell us that
  the higher the average grade, the lower the dropout rate. Usually
  students that have low grades are uninterested in the subjects which
  could be due to having chosen not the right program for them or that
  the way lectures and information is presented is uninteresting or
  outdated. Either way this can be improved. Increasing the possibility
  that the student has chosen the right program for him can be done by
  introducing more \enquote{open days} in higher education institutions
  and having more upfront information about what can be expected from
  programs. The overall lecture performance can be improved by taking
  more time to have up-to-date information presented and teachers having
  decent motivation of teaching students. This can be achieved by
  increasing teacher salaries and institutions having more control over
  teachers and information they present to students.
\end{enumerate}

All these metrics are still vulnerable to misrepresentation, but it is
inevitable given the freedom the universities enjoy in managing their
study programmes. Still, any manipulation of this metrics can only be
temporary and thus is also not in the best interest of the university.

\hypertarget{sec-eda}{%
\section{Exploratory Data Analysis}\label{sec-eda}}

\hypertarget{descriptive-statistics}{%
\subsection{Descriptive Statistics}\label{descriptive-statistics}}

As we have checked, the dataset does not have zero values, so there is
nothing to purge inside it. Later on, we get the basic descriptive
statistics, shown below in\\
Tables~\ref{tbl-descstat-1}, \ref{tbl-descstat-2}, \ref{tbl-descstat-3}, \ref{tbl-descstat-4}, \ref{tbl-descstat-5}, \ref{tbl-descstat-6}, \ref{tbl-descstat-7}

\hypertarget{tbl-descstat-1}{}
\begin{longtable}[]{@{}lrrrrr@{}}
\toprule\noalign{}
& Mari. stat. & Appl. mode. & Appl. orde. & Cour. & Dayt. atte. \\
\midrule\noalign{}
\endfirsthead
\toprule\noalign{}
& Mari. stat. & Appl. mode. & Appl. orde. & Cour. & Dayt. atte. \\
\midrule\noalign{}
\endhead
\bottomrule\noalign{}
\endlastfoot
count & 4424 & 4424 & 4424 & 4424 & 4424 \\
mean & 1.18 & 18.67 & 1.73 & 8856.64 & 0.89 \\
std & 0.61 & 17.48 & 1.31 & 2063.57 & 0.31 \\
min & 1 & 1 & 0 & 33 & 0 \\
25\% & 1 & 1 & 1 & 9085 & 1 \\
50\% & 1 & 17 & 1 & 9238 & 1 \\
75\% & 1 & 39 & 2 & 9556 & 1 \\
max & 6 & 57 & 9 & 9991 & 1 \\
\caption{\label{tbl-descstat-1}Descriptive statistics}\tabularnewline
\end{longtable}

\hypertarget{tbl-descstat-2}{}
\begin{longtable}[]{@{}lrrrrr@{}}
\toprule\noalign{}
& Prev. qual. & Prev. qual. (gra. & Naci. & Moth. qual. & Fath. qual. \\
\midrule\noalign{}
\endfirsthead
\toprule\noalign{}
& Prev. qual. & Prev. qual. (gra. & Naci. & Moth. qual. & Fath. qual. \\
\midrule\noalign{}
\endhead
\bottomrule\noalign{}
\endlastfoot
count & 4424 & 4424 & 4424 & 4424 & 4424 \\
mean & 4.58 & 132.61 & 1.87 & 19.56 & 22.28 \\
std & 10.22 & 13.19 & 6.91 & 15.6 & 15.34 \\
min & 1 & 95 & 1 & 1 & 1 \\
25\% & 1 & 125 & 1 & 2 & 3 \\
50\% & 1 & 133.1 & 1 & 19 & 19 \\
75\% & 1 & 140 & 1 & 37 & 37 \\
max & 43 & 190 & 109 & 44 & 44 \\
\caption{\label{tbl-descstat-2}Descriptive statistics
(cont'd)}\tabularnewline
\end{longtable}

\hypertarget{tbl-descstat-3}{}
\begin{longtable}[]{@{}lrrrrr@{}}
\toprule\noalign{}
& Moth. occu. & Fath. occu. & Admi. grad. & Disp. & Educ. spec. need. \\
\midrule\noalign{}
\endfirsthead
\toprule\noalign{}
& Moth. occu. & Fath. occu. & Admi. grad. & Disp. & Educ. spec. need. \\
\midrule\noalign{}
\endhead
\bottomrule\noalign{}
\endlastfoot
count & 4424 & 4424 & 4424 & 4424 & 4424 \\
mean & 10.96 & 11.03 & 126.98 & 0.55 & 0.01 \\
std & 26.42 & 25.26 & 14.48 & 0.5 & 0.11 \\
min & 0 & 0 & 95 & 0 & 0 \\
25\% & 4 & 4 & 117.9 & 0 & 0 \\
50\% & 5 & 7 & 126.1 & 1 & 0 \\
75\% & 9 & 9 & 134.8 & 1 & 0 \\
max & 194 & 195 & 190 & 1 & 1 \\
\caption{\label{tbl-descstat-3}Descriptive statistics
(cont'd)}\tabularnewline
\end{longtable}

\hypertarget{tbl-descstat-4}{}
\begin{longtable}[]{@{}llllll@{}}
\toprule\noalign{}
\endfirsthead
\endhead
\bottomrule\noalign{}
\endlastfoot
count & 4424 & 4424 & 4424 & 4424 & 4424 \\
mean & 0.11 & 0.88 & 0.35 & 0.25 & 23.27 \\
std & 0.32 & 0.32 & 0.48 & 0.43 & 7.59 \\
min & 0 & 0 & 0 & 0 & 17 \\
25\% & 0 & 1 & 0 & 0 & 19 \\
50\% & 0 & 1 & 0 & 0 & 20 \\
75\% & 0 & 1 & 1 & 0 & 25 \\
max & 1 & 1 & 1 & 1 & 70 \\
\caption{\label{tbl-descstat-4}Descriptive statistics (cont'd). Columns,
left-to-right: Debtor, Tuition fees up to date, Gender, Scholarship
holder, Age at enrollment}\tabularnewline
\end{longtable}

\hypertarget{tbl-descstat-5}{}
\begin{longtable}[]{@{}lllll@{}}
\toprule\noalign{}
\endfirsthead
\endhead
\bottomrule\noalign{}
\endlastfoot
count & 4424 & 4424 & 4424 & 4424 \\
mean & 0.02 & 0.71 & 6.27 & 8.3 \\
std & 0.16 & 2.36 & 2.48 & 4.18 \\
min & 0 & 0 & 0 & 0 \\
25\% & 0 & 0 & 5 & 6 \\
50\% & 0 & 0 & 6 & 8 \\
75\% & 0 & 0 & 7 & 10 \\
max & 1 & 20 & 26 & 45 \\
\caption{\label{tbl-descstat-5}Descriptive statistics (cont'd). Columns
International, Curricular units 1st sem (credited), Curricular units 1st
sem (enrolled), Curricular units 1st sem (evaluations).}\tabularnewline
\end{longtable}

\hypertarget{tbl-descstat-6}{}
\begin{longtable}[]{@{}lllll@{}}
\toprule\noalign{}
\endfirsthead
\endhead
\bottomrule\noalign{}
\endlastfoot
count & 4424 & 4424 & 4424 & 4424 \\
mean & 4.71 & 10.64 & 0.14 & 0.54 \\
std & 3.09 & 4.84 & 0.69 & 1.92 \\
min & 0 & 0 & 0 & 0 \\
25\% & 3 & 11 & 0 & 0 \\
50\% & 5 & 12.29 & 0 & 0 \\
75\% & 6 & 13.4 & 0 & 0 \\
max & 26 & 18.88 & 12 & 19 \\
\caption{\label{tbl-descstat-6}Descriptive statistics (cont'd). Columns
Curricular units 1st sem (approved), Curricular units 1st sem (grade),
Curricular units 1st sem (without evaluations), Curricular units 2nd sem
(credited)'}\tabularnewline
\end{longtable}

\hypertarget{tbl-descstat-7}{}
\begin{longtable}[]{@{}lllll@{}}
\toprule\noalign{}
\endfirsthead
\endhead
\bottomrule\noalign{}
\endlastfoot
count & 4424 & 4424 & 4424 & 4424 \\
mean & 6.23 & 8.06 & 4.44 & 10.23 \\
std & 2.2 & 3.95 & 3.01 & 5.21 \\
min & 0 & 0 & 0 & 0 \\
25\% & 5 & 6 & 2 & 10.75 \\
50\% & 6 & 8 & 5 & 12.2 \\
75\% & 7 & 10 & 6 & 13.33 \\
max & 23 & 33 & 20 & 18.57 \\
\caption{\label{tbl-descstat-7}Descriptive statistics (cont'd).
Curricular units 2nd sem (enrolled), Curricular units 2nd sem
(evaluations), Curricular units 2nd sem (approved), Curricular units 2nd
sem (grade)'}\tabularnewline
\end{longtable}

The students are from multiple countries, but the overwhelming majority
of the students are from Portugal. It would be interesting to see how
the students' admission grade depends on their previous qualification in
their home countries, but the samples are scarce. Many students from
abroad are from the Overseas Territories where it's more challenging to
get comparable education. However, they and inland Portugal students
were naturally given some exemptions, as the dataset states \footnote{Link
  to the dataset description:
  \url{https://archive.ics.uci.edu/dataset/697/predict+students+dropout+and+academic+success}}.
\footnote{Notably, the authors did not specify all categories of
  students even in the description to the dataset, so it can be regarded
  as one of \enquote*{issues} of the dataset that it can be challenging
  to fully interpret the feature store.}

For example, the students admitted per Ordance no. 854 \footnote{Link to
  the source document:
  \url{https://dre.tretas.org/dre/106607/portaria-854-B-99-de-4-de-outubro}}
were not required to demonstrate the proof of their fitness for studying
since their received a diploma in secondary education administered in
Portuguese (Angola, East Timor, Mozambique, Guinea Equatorial). Students
admitted per Ordnance no. 533 \footnote{Link to the source document:
  \url{https://dre.tretas.org/dre/104726/portaria-533-A-99-de-22-de-julho}}
were from another university in Portugal with overlapping courses
covered recently enough so they were not required to repeat them.
Finally, those admitted per Ordnance no. 612 \footnote{Link to the
  source document:
  \url{https://dre.tretas.org/dre/51542/portaria-612-93-de-29-de-junho}}
came from other countries but had comparable material in their studies
and so their points were recalculated with some amortization.

\hypertarget{data-visualizations}{%
\subsection{Data Visualizations}\label{data-visualizations}}

\begin{figure}

{\centering \includegraphics{report_AzadhdhinNedalYunisAlFraijat_files/figure-pdf/fig-education-output-1.png}

}

\caption{\label{fig-education}Relative graduation points for students
with different education backgrounds}

\end{figure}

Due to class imbalance , the variability for the Portuguese students is
much higher, and while the 3 categories (see Figure~\ref{fig-education})
with highest grades are natural, i. e. doctors, masters as higher
education, the 3rd is unintuitive (the 10 classes) and we tend to
explain it as self-selection and high correlation with other indicators
(those entering the university in the 10th grade are more motivated then
dwelling in schools in 11th and 12th grades).

Also, there is a drastic imbalance over yet another crucial factor: age.
Students of age are far less ubiquitous, can have far more incentives to
abandon studies and smaller potential for apprehension of material.
Indeed, this is clearly shown on the next graph \ref{fig-age-distr}

\begin{figure}

{\centering \includegraphics{report_AzadhdhinNedalYunisAlFraijat_files/figure-pdf/fig-age-distr-output-1.png}

}

\caption{\label{fig-age-distr}Distribution of age for dropout and
graduate student}

\end{figure}

Q. v. the sizes of the bins for dropout students differ far less than
the total size for the name of the student.

If the hypothesis about some external factors is correct , the target
variable should be much dependent on previous grades,\\
The datapoint cloud on Table~\ref{fig-points-distr}, however, shows that
this rule has a lot of exceptions.

\begin{figure}

{\centering \includegraphics{report_AzadhdhinNedalYunisAlFraijat_files/figure-pdf/fig-points-distr-output-1.png}

}

\caption{\label{fig-points-distr}Joint and marginal distributions of
current admission and previous qualification grades}

\end{figure}

We can draw the following observations:

\begin{itemize}
\item
  The distribution of admission grades is roughly normal with most
  students scoring between \textbf{\emph{120 and 160 marks}}.
\item
  The distribution of previous qualifications (grades) is also the same
  with most of them having grades in between \textbf{120 and 160}.
\item
  There is seen a positive correlation between admission grade and
  previous qualification grade indicating students with higher previous
  qualifications tend to have higher admission grades.
\end{itemize}

The visualizations above, however, were natural for the few quantitative
columns, which show the natural interconnection between the curricularly
accrued units in the 1st and in the 2nd year, which are in turn mostly
unrelated to the admission grade. This is understandable since the
grades are commonly based on the successfulness of the local program and
student's toil, while the students' backgrounds are commonly different
and this puts them into inequitable positions when passing the admission
exams.

In these previous graphs, we considered quantitative columns that are
more or less exogenous to the dataset (e. g. age and the previous
qualification grade are not influenced by the current grade of the
students).

However, the majority of columns of this dataset are qualitative and
they are at least partially endogenous as stem from the decisions during
the study and their consequences. For this, we need to propose a
mechanism of influence, then formulate and test a hypothesis via an
analysis of discriminate groups.

\begin{figure}

{\centering \includegraphics{./figs/reasons-1.jpg}

}

\caption{\label{fig-word-1}Student mobility and financial burden as
indicators and drivers of their motivation and impediments}

\end{figure}

We see on Figure~\ref{fig-word-1} that having debt is always a serious
impediment against studies because it gives wrong incentives towards
directly making money in the short run instead of focusing solely on
one's studies that could aid to make altogether greater money in the
long run.

\begin{figure}

{\centering \includegraphics{./figs/reasons2.jpg}

}

\caption{\label{fig-word-2}Student inter-university mobility and health
conditions proxies as indicators and drivers of their motivation towards
learning and impediments (Part 2)}

\end{figure}

We also consider the impact of scholarships and other compensations in
academic support, which should alleviate the complications associated
with adaptations in new environment.

\begin{figure}

{\centering \includegraphics{./figs/reasonsIII.jpg}

}

\caption{\label{fig-word-3}Marital status as distractor from studying}

\end{figure}

In different studies, it is quite common to compare the academic success
of a student with the academic successes of their parents as this has
both direct and indirect effects , s. a. i. e. both are connected to
welfare, but also it can be that there is another channel of knowledge
transmission to the younger generation.

\textbf{Observations :} * The bar chart shows that mother's occupation
is quite influential. This influence is greater the pa's due to
traditional effect, and we distinctly see that students whose mothers
are \enquote*{white collars} dropout significantly more rarely than
those whose mothers are more engaged in physical labor.

\begin{itemize}
\tightlist
\item
  This also may suggest the mother's occupation can influence student
  retention, emphasizing the need for financial support and family
  engagement.
\end{itemize}

\hypertarget{data-correlation-table-quantitative-columns-only}{%
\subsection{Data correlation table (quantitative columns
only)}\label{data-correlation-table-quantitative-columns-only}}

\begin{figure}

{\centering \includegraphics{report_AzadhdhinNedalYunisAlFraijat_files/figure-pdf/fig-correlation-output-1.png}

}

\caption{\label{fig-correlation}Data correlation table (quantitative
columns are represented only since there to compute true correlations
between quantitative columns it is necessary to OHE-encode them, which
would burst no. of columns to many thousands, but the values of the
correlations will be statistically insignificant due to low cardinality
of 90\% of classes)}

\end{figure}

As visible on Figure~\ref{fig-correlation}, there are strong positive
correlation among the curricular units for 1st and 2nd sem that could
suggest that students that are doing well in the first sem will
certainly do well in the upcoming semester. This could suggest that
students with lower curricular units have a higher chance of dropping
out due to their academic preparedness and failing to meet the
requirements and expectations of the study program.

In the remaining part, we examine the correlations of purely endogenous
temporal variables. This does not give a scoop about the source of
causation and is not a good predictor, but exhibits an analysis of
autocorrelation inside the quasi-temporal data.

\begin{figure}

{\centering \includegraphics{report_AzadhdhinNedalYunisAlFraijat_files/figure-pdf/fig-cur-units-output-1.png}

}

\caption{\label{fig-cur-units}Vizualization for curricular successively
approved units}

\end{figure}

We can see that the points for the 1st semester and 2nd semester are
correlated which shows that are one's marks are primary drivers of
success and exhibit sizeable correlations

\begin{figure}

{\centering \includegraphics{report_AzadhdhinNedalYunisAlFraijat_files/figure-pdf/fig-cur-units-enrolled-output-1.png}

}

\caption{\label{fig-cur-units-enrolled}Vizualization for curricular
successively enrolled units}

\end{figure}

\begin{figure}

{\centering \includegraphics{report_AzadhdhinNedalYunisAlFraijat_files/figure-pdf/fig-cur-grade-output-1.png}

}

\caption{\label{fig-cur-grade}Vizualization of curricular units for the
1st semester}

\end{figure}

\hypertarget{unmentioned-data-issues}{%
\subsection{Unmentioned data issues}\label{unmentioned-data-issues}}

While the dataset does not have empty columns , there is some auxiliary
data that could be very appropriate to have together with it. For
example, we have data for years that are many in today's changing
education environment, but there is no open data as proxy for the
conditions in which the students have to study and which influence their
dropout ratios (causing affection or not). Apart from data on
schorlarships, any transitions and their effect (even though hapenned
during reforms in Portugal education field in 2000s-10s) are downplayed.
Another significant challenge are extraneous events that fundamentally
influence the life of student, such as permanent homecare of one of the
relatives, or a new job, or transfer to another university which is more
preferable. In each of three and arguably more cases, no known factors
can stop a person from leaving, and any model on this dataset would
fail. From the societal point of view, these leaves should not be
pejorized as dropouts, since this makes students more felicitous. Since
employment and civil status could be retrieved from Social Services, it
would be nice if we checked our model against \enquote*{undesirable}
leaves only.

\hypertarget{sec-data-mining}{%
\section{Data Mining (Analysis)}\label{sec-data-mining}}

\hypertarget{ml-pipeline-design}{%
\subsection{ML pipeline design}\label{ml-pipeline-design}}

Now, we need to define our data mining strategy.

In the matrix \ref{fig-correlation} for correlations, we already see
high correlations between many values. Hence, if we (certainly we
should) also consider categorical variables in our data mining analysis,
we must reduce the number of variables because the true dimensionality
of the initial space is too high and virtually all ways of embedding and
distance calculation are too costly and prohibitive given a relatively
small amount of datapoints in this dataset.

In EDA Section~\ref{sec-eda}, we already stressed the issue of class
imbalance. To at least partially recompense for this, we need to perform
SMOTE augmentation of scarcer classes. After dataset is SMOTE-amplified,
our common step would be to dispose of multicollinear columns.\\
High dimensionality prevents intuitive DBSCAN threshold setting and some
inferior algorithms as TSNE. Hence, although not considering distal
non-linear data structures, we reduce the dimensionality via principal
component analysis. After we perform the PCA, we perform Yeo-Johnson
power transform to make dedimensionalized data look more like
multivariate normal distribution (although some features like schooling
points already have near-normal distribution, joint distribution
normality for a vector linear combinations is not guaranteed without
pretransforming). After the distribution of the data has become closer
to Normal Multivariate, it is now possible to apply the outlier
detection method. Since the linear PCA transform has alienated initial
neighbours and thwisted the distances, it is not correct to apply
distance-based outlier detection methods such as outlier kNN, LOF, and
others. Instead, we would apply such method as Isolation Forest, because
its criterion is invariant by linear transformation, and now, given the
unimodality of the distribution, it is straghtforward to locate the
outliers based on their secludibility.

Then several estimators from various standard families are independently
trained and va fine-tuned, by mean F1 measure, the model that is most
precise in predicting the outcome is rendered. We confer some other
renowned metrics for the best model . In total, we consider 3 different
types of models for classification (logical, linear separative, linear
generative), implying algorithms of various \enquote*{difficulty}
levels, including 3 boostings: LGBoosting, XGBoosting (gbc), Catboost,
Random Forest, a Decision Tree as well as linear discriminate analysis
and logistic regression. We test models against each other and against
dummy classifier. The results of the best models are given in
leaderboard below .

\hypertarget{data-mining-application}{%
\subsection{Data mining application}\label{data-mining-application}}

\begin{table}

\begin{minipage}[t]{\linewidth}

{\centering 

}

\end{minipage}%
\newline
\begin{minipage}[t]{\linewidth}

{\centering 

\begin{longtable}[]{@{}lll@{}}
\toprule\noalign{}
& & \\
& & \\
\midrule\noalign{}
\endhead
\bottomrule\noalign{}
\endlastfoot
Initiated & . . . . . . . . . . . . . . . . . . & 13:48:33 \\
Status & . . . . . . . . . . . . . . . . . . & Selecting Estimator \\
Estimator & . . . . . . . . . . . . . . . . . . & Logistic Regression \\
\end{longtable}

}

\end{minipage}%
\newline
\begin{minipage}[t]{\linewidth}

{\centering 

}

\end{minipage}%
\newline
\begin{minipage}[t]{\linewidth}

{\centering 

}

\end{minipage}%
\newline
\begin{minipage}[t]{\linewidth}

{\centering 

\begin{longtable}[]{@{}llllllllll@{}}
\toprule\noalign{}
~ & Model & Accuracy & AUC & Recall & Prec. & F1 & Kappa & MCC & TT
(Sec) \\
\midrule\noalign{}
\endfirsthead
\toprule\noalign{}
~ & Model & Accuracy & AUC & Recall & Prec. & F1 & Kappa & MCC & TT
(Sec) \\
\midrule\noalign{}
\endhead
\bottomrule\noalign{}
\endlastfoot
catboost & CatBoost Classifier & 0.7035 & 0.8558 & 0.7035 & 0.7213 &
0.7097 & 0.5235 & 0.5256 & 1.6310 \\
gbc & Gradient Boosting Classifier & 0.6964 & 0.8609 & 0.6964 & 0.7332 &
0.7086 & 0.5197 & 0.5249 & 0.7690 \\
rf & Random Forest Classifier & 0.7019 & 0.8570 & 0.7019 & 0.7214 &
0.7084 & 0.5207 & 0.5230 & 0.2150 \\
ridge & Ridge Classifier & 0.6977 & 0.0000 & 0.6977 & 0.7214 & 0.7048 &
0.5136 & 0.5170 & 0.0890 \\
lightgbm & Light Gradient Boosting Machine & 0.6987 & 0.8533 & 0.6987 &
0.7155 & 0.7045 & 0.5152 & 0.5172 & 0.7850 \\
lda & Linear Discriminant Analysis & 0.6890 & 0.8473 & 0.6890 & 0.7397 &
0.7037 & 0.5107 & 0.5191 & 0.0780 \\
dt & Decision Tree Classifier & 0.6124 & 0.7074 & 0.6124 & 0.6386 &
0.6226 & 0.3879 & 0.3904 & 0.0910 \\
lr & Logistic Regression & 0.3621 & 0.5124 & 0.3621 & 0.2756 & 0.3035 &
0.0460 & 0.0541 & 0.7850 \\
dummy & Dummy Classifier & 0.3211 & 0.5000 & 0.3211 & 0.1031 & 0.1561 &
0.0000 & 0.0000 & 0.0840 \\
\caption{}\label{T_60247}\tabularnewline
\end{longtable}

}

\end{minipage}%
\newline
\begin{minipage}[t]{\linewidth}

{\centering 

\begin{verbatim}
Processing:   0%|          | 0/41 [00:00<?, ?it/s]
\end{verbatim}

}

\end{minipage}%

\caption{Results of fitting estimators of different families}

\end{table}

Thus, the best model by F1 measure is CatBoostClassifier, which is
renowned for scoring fairly well on low magnitude tabular data, while
ordinary GBTC is the most second to prime and the most robust one,
featuring best conventional recall, accuracy, and AUC metrics. The
Figure~\ref{fig-rocauc} figure shows the ROC AUC plot for the model

\begin{figure}

{\centering \includegraphics{report_AzadhdhinNedalYunisAlFraijat_files/figure-pdf/fig-rocauc-output-1.png}

}

\caption{\label{fig-rocauc}Stylized ROC curves for the Catboost Model
(for different classes and on average)}

\end{figure}

Our CatBoost parameters were already opinionated, with good rule of
thumbs, however, it is essential that we ensure that we retrieve
everything from this model, and so we performed a tuning of the best
model with Tree Parzen Estimator (TPE) tuner. This did not, however,
yield a robustly better model within a sensible number of iterations
(the results are available upon request). Next, we return to the models
that were not much worse than our top model in the leaderboard. Random
Forest that scored well is known to make low variance and GBTC
classifier makes lower bias, so it can be benefitial to combine them. We
consider 2 meta-model settings: the simple averager and the Logistic
Regression. In our experiments, Logistic Regression did not bring any
sizeable improvement (See Figure~\ref{fig-stacker}) but the averager
improved average F1 by 0.03 points (See Figure~\ref{fig-blender}).

On Figure~\ref{fig-blender} there is the figure that exhibits the ROC
curve for the blender model.

\begin{figure}

{\centering \includegraphics{report_AzadhdhinNedalYunisAlFraijat_files/figure-pdf/fig-blender-output-1.png}

}

\caption{\label{fig-blender}Blender model for the Top 3 models from the
leaderboard}

\end{figure}

On Figure~\ref{fig-stacker}, there is the stacker model ROCAUC to
improve the F1 measure. It plays a little better than without any
supermodel, but this is not as high as with simple blending

\begin{figure}

{\centering \includegraphics{report_AzadhdhinNedalYunisAlFraijat_files/figure-pdf/fig-stacker-output-1.png}

}

\caption{\label{fig-stacker}Stacker model for the Top 3 models from the
leaderboard with logistic regression as the supermodel}

\end{figure}

\begin{figure}

{\centering \includegraphics{report_AzadhdhinNedalYunisAlFraijat_files/figure-pdf/fig-explanation-output-1.png}

}

\caption{\label{fig-explanation}The average contribution of different
components of the Principal Component Decomposition on the target
variable (by label)}

\end{figure}

Currently, we have a prediction model, but its decisions are not
transparent enough, because it is a stacked boosting of sequences of
conditions on sophisticated linear combinations of several dozens of
source variables. To add explainability, we decide to compute the input
of each source variable via the following procedure. First, we examine
the behaviour of the model on production data with and without several
features, and thus compute the Shapley values for the model by its
components, presented on Figure~\ref{fig-explanation}. Then, we infer
the coefficients for variable decomposition for the components (See
Figure~\ref{fig-decomposition} for exemplary distribution of weights of
the 3rd component) and thus multiplying the weights by modulo of
interstitial power function transform we can ultimately obtain a
breakdown for variables for the decision of our best model for each
particular case.

\begin{figure}

{\centering \includegraphics{report_AzadhdhinNedalYunisAlFraijat_files/figure-pdf/fig-decomposition-output-1.png}

}

\caption{\label{fig-decomposition}Decomposition of the weights for the
3rd component}

\end{figure}

However, while all top models in leaderboard and in the stacking models
demonstrate significant improvement over a dummy classifier and other
simplistic models such as Logistic Regression, the scores still a lot to
be desired, which indicates that reduction of dimensionality, which is
inevitable under given class imbalance, has come at a price of variance
loss, or, alternatively, all the covariates do not explain sufficiently
well the outcome of studies: in academic success, as in life, a lot
depends on the proper characteristics of a person which are difficult to
elicit and much is undetermined. Partially, such a moderate result can
be exaplained by the multiclass nature of the test: the ROCAUC plot on
Figure~\ref{fig-rocauc} shows, AUC is worse on class 1, i. e.
\enquote{Enrolled}. These enrolled people have include both pending
dropouters and diligent scholars, and the core feature of the model is
that it can tell two other classes apart better, with ROCAUC 0.9. Thus,
this model can generally be used as an indicator of a soon dropout of a
student which the universities can be wary of, undertaking the
corrective action or allocating the stipends.

\hypertarget{results}{%
\section{Results}\label{results}}

With this analysis, we have some valuable insights about some crucial
factors like academic support, socioeconomic factors, age, previous
qualifications, and others that play a significant role in student
retention. We have built a model on all available data to predict the
dropout rate of students, that is quite lightweight, production-ready
and thus can easily be incorporated into some advisory system.

The observed patterns imply a lot to stress and decide in the lives of
students and their associates. First, we strive to insentivize parents
to improve their labour efficiency and pursue greater carreer so that
ultimately they could dedicate more time to their children's education,
and proactively stir their self-propelled interest. Additionally, we
could provide financial assistance to those who are struggling to pay
with if this is contemporaneous with a significant degradation in their
university marks, as this subrogates the stimuli for a person in an age
where they are most perceptive to knowledge and is a good predictor of a
dropout. Also importantly, we could teach the students, especially going
on their second studies, that it is quite unlikely that they are going
to get high grades or exit the university without proper time management
and confirmation that they assign top priority to their studies. They
are also advised to make that clear to all their relatives and
stakeholders who might underestimate the effects of such a change.
Although this could result in a reduction of enthusiastic entrants, this
would increase at least the KPI of retention and arguably also increase
the KPI on number of diplomas issued, because with fewer but more
motivated students the university would have more time to dedicate to
most obstinate pending alumni.

Addressing these factors carefully can effectively lead to dropout rates
reduction and improve overall student outcomes


\printbibliography


\end{document}
